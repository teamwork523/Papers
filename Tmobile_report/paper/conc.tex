\section{Conclusions}
\label{sec:conc}

From the RRC state inference model, I evaluated the latency across each RRC state and state transitions. I observed the extra latency in the FACH state and FACH promotion transitions. Utilizing the QxDM monitoring tool, we have a better visibility over RLC layer traffic information. I wrote a QxDM log parser and analyzer to cross mapping the transport layer and data link layer information, and identified the root cause of abnormal delays cause by channel contention and imperfection in the RLC protocol. I proposed a RLC \textit{Fast Re-Tx} mechanism to actively response to the PDU loss, and evaluated the delay cost-benefit over real-time traces. The latency reduction over FACH and FACH promotion state is up to \textit{35.69\%}, and it could also reduce the overall latency by \textit{2.66\%}. Therefore, the RLC \textit{Fast Re-Tx} mechanism could enhance the user mobile experience by introducing less delays, especially during the initial states of data transmission.
