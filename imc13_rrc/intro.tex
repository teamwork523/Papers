\section{Introduction}
%\renewcommand{\labelitemii}{$\star$}
Cellular network technologies, such as 3G \textit{Universal Mobile Telecommunications System} (UMTS) and 4G \textit{Long Term Evolution} (LTE), require that devices transition between various RRC states based on the network traffic patterns of individual mobile clients.  These states have different performance and energy consumption tradeoffs, as well as different state transition delays.  Using high-power RRC states only when necessary allows users to experience good network performance on resource-constrained mobile devices. Although the RRC states are defined by \textit{3rd Generation Partnership Project} (3GPP)~\cite{spec-3G-RRC, spec-LTE-RRC}, many aspects of the RRC state machine, such as timers for transitioning between states, are implementation or configuration-specific, differing by device model, network operator and location.  The real-world deployment of RRC state machines and the impact on performance are not well-understood. 

A better understanding of RRC state behavior would be beneficial for many parties.  Cellular network operators would be interested in determining how devices on their networks perform and how performance can be improved.  Device manufacturers would be interested in detecting device-dependent effects --- device implementation details and features such as Fast Dormancy~\cite{fast_dormancy} can mean different devices perform differently. Developers of major applications might be interested in understanding how network behavior can impact application performance, as work has been done to show that application behavior can decrease RRC-related performance issues~\cite{aro}.  Finally, researchers studying protocol optimization would find a better understanding of RRC state behavior valuable. 

In this paper, we examine how RRC states are implemented, behave and perform in real-world implementations. We present a new approach to inferring RRC state transitions that addresses limitations in previous work which prevent inference from being done in non-ideal network conditions or by non-expert users. In particular, we focus on detecting performance anomalies caused by RLC-layer \textit{transient states}, which can have a significant performance impact. We perform a study of RRC state implementations and performance on 16 network operators from 9 countries,  and use RLC layer analysis to understand the root causes of behavior and performance trends discovered. We present three main categories of contributions: %. First, we present several methodological contributions which allow us to perform measurements that lead to a deeper understanding of RRC behavior. Second, we present our findings on previously unknown performance problems and phenomena. Third, we suggest some potential performance optimizations. 

%\begin{enumerate}[noitemsep,topsep=0pt,parsep=0pt,partopsep=0pt]
%[noitemsep,topsep=0pt,parsep=0pt,partopsep=0pt]
%\item
\noindent\labelitemi\indent \textbf{Methodological contributions.} We have created two types of tools. First, we designed a novel generic RRC state model inference method that is robust to poor network conditions and interfering background traffic on the mobile device. It is implemented in an Android application and is designed to be usable out of the box by non expert users.  We focus primarily on inferring the demotion timers of RRC states as well as the performance associated with each state. We hope to release this tool as open-source software. Second, we determined methods to effectively make use of RLC-layer traces from \textit{Qualcomm eXtensible Diagnostic Monitor} (QxDM). We developed a novel cross-layer mapping mechanism that correlates transport layer packets with data layer link \textit{protocol data units} (PDUs, the smallest data transmission unit in RLC). It can successfully map 99.8\% of transport-layer packets to PDUs. This allows the root causes of TCP and UDP performance and packet loss issues to be accurately identified at the RLC layer.

%\item
\noindent\labelitemi\indent  \textbf{New Findings on the Impact of RRC State on Performance.} We analyze the RRC state implementation and associated performance characteristics for a number of network operators, showing that our methods are useful in understanding RRC states in the real world. We present several new findings that show there are significant undetected performance problems in networks today. At the RLC layer, we show that there are frequent RLC-layer retransmission delays surrounding \textit{transient states}, especially during state transitions to and from FACH. 

%\item 
\noindent\labelitemi\indent \textbf{Proposed Solution to Performance Problems Found.} We propose an improvement to how RLC currently behaves based on our improved understanding of RLC-level behavior. To reduce potentially unnecessary retransmissions that occur during poorly-performing FACH-related transient states, we propose an RLC \emph{Fast Re-Tx} mechanism that actively responds to the RLC PDU loss signals. By simulating this fast retransmission mechanism using real QxDM traces that provide visibility of RLC traffic behavior, we find that our proposed mechanism could reduce RLC latency by up to 35.69\% when these poorly-performing FACH states occur. In other states, the latency is still reduced by 2.66\% on average.

% The possible latency overhead occurs when we fail to speculate the RLC retransmission, but the overall latency could still be reduced by 2.66\%.
%\end{enumerate}


%%	\begin{itemize}
%		\item
%			\textbf{Methodology:} Based on state of art RRC state inference method, we designed a generic RRC state model generation mechanism to effectively reduce the effect of unstable wireless channel, and produce finer RRC state machine using classic statistical method. In additional, we develop the first cross layer mapping mechanism in the empirical study to correlate the transport layer packets with the data layer link PDUs, so that we could accurately identify the TCP/UDP performance issue inside the RLC layer. We could map 99.8\% of the transport layer packets to the corresponding RLC layer PDUs.
%		\item
%			\textbf{Observation:} We observe the abnormal large FACH state latency issue through our RRC inference measurement. After a world wide scale deployment of RRC state model generation measurement, we are able to discover the device dependent and network specific abnormal FACH state behaviors. We also observe the frequent RLC layer retransmission and retransmission delay over initial FACH state and FACH promotion transition, which contributes a lot to the transport layer latency.
%		\item	
%			\textbf{Implications:} Due the abnormal initial FACH state and FACH latency behavior, the application could batch the data transmission to reduce the possible number of transmission over FACH. To address the RLC retransmission delay issue, we propose a RLC \emph{Fast Re-Tx} mechanism to active response to the RLC PDU loss signals. By simulate the fast retransmission mechanism in the real trace QxDM, we apply cost-benefit analysis and found that \emph{Fast Re-Tx} could reduce RLC latency up to 35.69\% over FACH states.
%	\end{itemize}
