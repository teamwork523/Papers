\section{Related Work}
\label{sec:related}

\textbf{Cross-layer Analysis:} ARO~\cite{aro} is passive measurement tool that correlates application layer, transport layer, and RRC layer to identify the unnecessary energy consumption due to improper application design. ARO only refers RRC state using network latency measurement results due to lack of accessibility to ground true lower layer information. \textit{TransLayer} not only has access to the lower control plane information, but also the data plane. That benefits us from performing fine-grained lower information collection and analysis, and we are more confident about the accuracy of cross-layer mapping results. 

Rilanalyzer~\cite{rilanalyzer} enables ground truth RRC information on Intel/Infineon XGold chipsets with root access, and it does not requires to connect to a personal computer. Although RRC state allocation could affect the energy consumption on the devices, it has a much less impact on the device performance. QxDM provides much detailed lower layer information that allows us to extract tons of lower layer features to pinpoint the root causes of the performance issues.

The survey~\cite{cross-layer.wireless} summarized the cross-layer design for wireless network, and highlighted several innovative design for non-layered network topology. It indicates that many performance problems come from the blocked communication mechanism between different network layers. The performance issue from a single layer could propagate to others due to not inform other layers in time. In our study, we would like to support that introducing communication between layers could potentially benefit the overall performance in the cellular context.

\textbf{Quality of Experiences:} In recent years, people started to shift focus from ``best effort" level QoS performance improvement study to a more realistic QoE user-oriented analysis. Moorsel defined quality of experience as the combination of reliability with performance, or in his words ``performability"~\cite{qoe.define}. Ickin, et al., proposed various QoE metrics for mobile applications from a Human Computer Interaction (HCI) perspective~\cite{qoe.application}, which covers application performance, battery, phone features, app and data cost, user's routine, and user's lifestyle.

~\cite{user.engagement.impact} studied the impact of video quality on user engagement, and used measurement results to approve that the QoS performance metrics do not linearly related to user engagement metrics.~\cite{qoe.predict} proposed a predictive model that utilized actionable performance metrics to improve the user engagement using decision tree from machine learning. Those study collect user feedback information passively.Our active real-time feedback collection mechanism is actually the opposite approach. \textit{TransLayer} focuses on locating the ground truth information about bad user engagement.